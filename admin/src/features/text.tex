Service axios lay du lieu tu backend sang react

// createAsyncThunk được sử dụng để tạo một thunk không đồng bộ ( fetchDataThunk) đóng gói logic để tìm nạp dữ liệu từ API.

// Trường extraReducerstrong createSliceđược sử dụng để xử lý các hành động này và cập nhật trạng thái Redux tương ứng.

// login.pending: Điều này được gửi đi khi yêu cầu  đang chờ xử lý. Nó đặt isLoadingthành đúng.

// login.fulfilled: Điều này được gửi đi khi yêu cầu thành công. Nó cập nhật trạng thái bằng cách đặt isLoadingthành false và isSuccess: true

// login.rejected: Điều này được gửi đi khi yêu cầu bị từ chối (thất bại)

export const login = createAsyncThunk
  // Tên action
  'user/login',
  - thunkAPI: Nó nhận được thunkAPInhư một đối số.
% /////////////////////////////////////
Cách viết API":
1: Lấy Api bằng Axios qua đường link get, post,... từ bạckend sang => Viết ở folder Service
2: Lấy api backend sang Slice để xử lý dữ liệu 
+ Xử lý bất đồng bộ và trả ra kết qủa
+ Tạo State ban đầu
+ Nếu thành công sẽ gắn payload cho products
3: Đưa sang store Reducer sẽ lấy được tất cả trên project
4: đưa lên client 
+ Gửi hành động lên bằng dispatch
+ truy xuất bằng hàm useSelect để lấy dữ liệu 

% //////////////////////
-Yup: Xác định quy tắc xác thực cho các trường biểu mẫu của bạn
Ví dụ, khi bạn khai báo onChange={formik.handleChange},handleChange của Formik để cập nhật trạng thái của form. 
Điều này giúp Formik theo dõi giá trị của mỗi trường và thực hiện các kiểm tra hợp lệ.
- onSubmit={formik.handleSubmit}: trước khi fỏrm được gửi, Sử dụng handleSubmir để xử lý các sự kiện bên trong xem nó đã đúng chưa
onBlur={formik.handleBlur}: Tóm lại là hàm này khi người dùng nhấn vào nó sẽ kiểm tra xem có hợp lệ hay không qua schema validate(yup)
value={formik.values.fieldName} = nó tương ứng với trường trong state để đồng bộ hoá
